\chapter{Einleitung}

Im Rahmen des Seminars 'Ethische Fragen der Informatik' an der Humboldt-Universität zu Berlin, sollte sich der Kurs mit verschiedenen aktuellen Themen auseinandersetzen. Diese Gruppe hat einen Vortrag zu sogenannten 'Biased Algorithems' ausgearbeitet, der im Nachfolgenden nochmals verschriftlich wurde.

\section{Motivation}

Aufgrund von häufigeren Berichten in Onlinemagazinen und Social-Media-Plattformen, wird das Thema 'Biased Algorithmens' immer präsenter in unserer Gesellschaft. Vor Allem, da Algorithmen immer mehr Probleme in unserem Alltag lösen und somit einen zunhemend größeren Einfluss auf unser soziales Leben nehmen.
Gerade als Student-in der Informatik stellt sich einem die Frage, welche Verantwortung man selbst im späteren Berufsleben beim Entwickeln von Software hat. Es stellt sich also die Frage, was ist ein 'Biased Algorithm' und wie kann man dies bei der Entwicklung vermeiden.



\section{Begriffserklärung}


\subsection{Algorithmus}
Da es verschiedene Definitionen des Wortes Algorithmus gibt, wurde der Versuch unternommen im Rahmen dieser Ausarbeitung lediglich Kriterien zu benennen die einen Algorithmus im Sinne des Themas von anderen Begriffen wie Software, Framework oder künstlicher Intelligenz unterscheiden. Dabei wurden 5 Kriterien festgelegt:
\begin{itemize}
	\item Maschinenlesbarkeit / Maschinenausführbarkeit
	\item Lösung eines konkreten Problems
	\item Berstehend aus einer endlichen Menge an Anweisungen
	\item Endliche Verarbeitungszeit
	\item Gleiche Eingabe erzeugt gleiche Ausgabe
\end{itemize}
Insbesondere das letzte Kriterium wurde gewählt um sich bewusst von Thema künstlicher Intelligenz abzusetzen.

% * <Alena Schemmert> 17:05:29 23 Sep 2019 UTC+0200:
% Noch offen
\subsection{Biased Algorithm}
'Biased' ist ein englisches Wort, für welches es viele Übersetzungen gibt. Dazu gehören Verzerrt, Parteiisch, Voreingenommen, Tendenziös, Uneinheitlich. Die treffendste Übersetzung ist verzerrt, da parteiisch, voreingenommen oder tendenziös einem Algorithmus unterstellen würden, er hätte eine eigne Meinung. Uneinheitlich widerspricht der allgemeinen Definition eines Algorithmus, da dieser ja deterministisch sein soll.  
