\chapter{Zusammenfassung}

\section{Zusammenfassung}

In dieser Arbeit wurde verschiedene Arten von verzerrten Algorithmen beschrieben. Hierunter fallen solche Algorithmen, welche unter Umständen unfaire oder nicht nachvollziehbare Ergebnisse liefern. Ein schlechter Entqurf oder eine fehlerhafte Programmierung beziehungsweise Auswahl an Beispieldaten kann zu Diskriminierung führen. In dem Abschnitt Allgemeine Bedenken werden einige Anhaltspunkte gegeben, anhand derer sich erkennen lässt, dass ein Algorithmus verzerrte Ergebnisse liefern könnte.  
    Uneindeutige, Undurchsichtige und Fehlgeleitete Anhaltspunkte führen zu Unfairen Ergebnissen und Transformativen Einflüssen in unserer Gesellschaft.
    Wenn diese Faktoren auftreten leidet die Nachvollziehbarkeit der Ergebnisse der Algorithmen und die der Software, die diese benutzt.

Im letzten Abschnitt wurden verschieden Beispiele für verzerrte Algorithmen gegeben. Da diese nicht alle den gleichen Einfluss auf die Gesellschaft haben, lassen sie sich die Beispiele f
grob in drei Problemfelder einteilen: Entscheidungsalgorithmen, Empfehlungalgorithmen und Life-Style-Software. 

\section{Fazit}

% * <Alena Schemmert> 17:13:55 23 Sep 2019 UTC+0200:
% Muss noch geschrieben werden


\section{Ausblick}
Es gibt Vorschläge zu Maßnahmen und Methoden zur Vermeidung von fahrlässiger Verzerrung von Algorithmen schon während der Software-Entwicklung. Hierzu gehören verschiedene Herangehensweisen, die beispielsweise eine kontrollierte Verzerrung der Beispieldaten oder Trainingsdaten nutzen, um von Anfang an eine mögliche Diskriminierung im Auge behalten zu können. Ebenfalls könnten für Klassifikations-Modelle verschiedene Anti-Diskriminierungs-Kriterien erstellt und immer wieder auf neutralität überprüft werden. Hilfreich bei dieser Herangehensweise sind verschiedene Entwicklungsmethoden, agile Methoden oder Test-Driven-Developement. \newline 
Die EU-Kommission beschäftigt sich ebenfalls mit dieser Problematik und hat eigens dafür eine eine unabhängige Expertengruppe berufen, welche nun eine ethische Leitlinie für die Entwicklung vertrauenswürdiger Künstlicher Intelligenz entworfen hat. (Stand Juni 2019) \cite{europe2019} Inwieweit diese Leitlinie auch auf andere Algorithmen, die nicht dem maschinellen Lernen entspringen, angewendet werden wird, bleibt noch offen. 
\newline
In der heutigen Gesellschaft haben Computer und Algorithmen mittlerweile einen echten Einfluss auf unsere Lebensrealität und dieser wird sich in den nächsten Jahren mit Sicherheit noch verstärken. Entsprechend dazu vergrößert sich die Verantwortung der Entwickler und man darf hoffen, dass die Problematik der ethischen Korrektheit wieterhin behandelt wird.


